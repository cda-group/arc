\begin{abstract}

% What to include in the abstract:
% * The problem description of the paper
%   - Data analysis is becoming more complicated from a programming standpoint.
%   - Existing systems are not flexible enough to support new requirements.
% * The main idea and contributions of the paper
%   - The main idea is to use a programming language to solve the problem.
% * The main results of the paper
% * The main conclusions of the paper

Data analytics pipelines are becoming increasingly more complicated due to the growing number of requirements imposed by data science. Not only must data be processed and analyzed scalably with respect to its volume and velocity, but also intricately by involving many different data types. Arc-Lang is a programming language for data analytics that supports data parallel operations over multiple data types including data streams and data frames. In this paper we give a formal definition of Arc-Lang along with examples of its applications. We describe how Arc-Lang programs are translated into Rust using the MLIR compilation framework and then deployed on a distributed system. We compare Arc-Lang to different high-level DSLs for data intensive computing based on their problem domains, programming models, and implementations. Finally, we discuss future trends and open research questions in the area of DSLs for data intensive computing.

\end{abstract}
